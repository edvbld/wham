\documentclass[11pt,oneside,a4paper]{article}
\usepackage{amsmath}
\usepackage{amsfonts}
\usepackage{stmaryrd}
\usepackage{listings}

\lstset{
  language = Java,
  breaklines = true,
  captionpos = b,
  basicstyle = \footnotesize,
  frame = leftline,
  morekeywords = {skip,then},
}

\newenvironment{changemargin}[2]{%
\begin{list}{}{%
\setlength{\topsep}{0pt}%
\setlength{\leftmargin}{#1}% \setlength{\rightmargin}{#2}%
\setlength{\listparindent}{\parindent}%
\setlength{\itemindent}{\parindent}%
\setlength{\parsep}{\parskip}%
}%
\item[]}{\end{list}}

\newcommand{\SExp}[2]{\mathcal{#1}\llbracket #2 \rrbracket}
\newcommand{\AExp}[2]{\SExp{A}{#1}(#2)}
\newcommand{\BExp}[2]{\SExp{B}{#1}(#2)}
\newcommand{\AMIns}[1]{\textsc{#1}}
\newcommand{\AMConf}[3]{\langle #1, #2, #3 \rangle}
\newcommand{\AMArrow}{\: &\triangleright&\:\;}
\newcommand{\sign}{\textbf{Sign}_{\bot}}
\newcommand{\TT}{\textbf{TT}_{\bot}}
\newcommand{\abs}[2]{\textbf{abs}_{\,#1_{\bot}} #2 }
\begin{document}
\title{Report for Lab Assignment 2, DD2457 Program Semantics and Analysis}
\author{Erik Helin \& Oskar Arvidsson}
\date{\today}
\maketitle
\section*{Operational semantics for AM}
\begin{align*}
&\AMConf{\AMIns{push-}n:c}{e}{s} \AMArrow 
 \AMConf{c}{\abs{Z}{\SExp{N}{n}}:e}{s} \\
&\AMConf{\AMIns{add}:c}{v_1:v_2:e}{s} \AMArrow
 \AMConf{c}{v_1 +_{SE} v_2:e}{s}& \text{ if } v_1, v_2 \in \:& \sign \\
&\AMConf{\AMIns{sub}:c}{v_1:v_2:e}{s} \AMArrow
 \AMConf{c}{v_1 -_{SE} v_2:e}{s}& \text{ if } v_1, v_2 \in \:& \sign \\
&\AMConf{\AMIns{mul}:c}{v_1:v_2:e}{s} \AMArrow
 \AMConf{c}{v_1 \star_{SE} v_2:e}{s}& \text{ if } v_1, v_2 \in \:& \sign \\
&\AMConf{\AMIns{div}:c}{v_1:v_2:e}{s} \AMArrow
 \AMConf{c}{v_1 \: /_{SE}\: v_2:e}{s}& \text{ if } v_1, v_2 \in \:& \sign \\
&\AMConf{\AMIns{true}:c}{e}{s} \AMArrow
 \AMConf{c}{\abs{T}{\textbf{tt}}:e}{s}& \\
&\AMConf{\AMIns{false}:c}{e}{s} \AMArrow
 \AMConf{c}{\abs{T}{\textbf{ff}}:e}{s}& \\
&\AMConf{\AMIns{eq}:c}{v_1:v_2:e}{s} \AMArrow
 \AMConf{c}{v_1 =_{SE} v_2:e}{s}& \text{ if } v_1, v_2 \in \:& \sign \\
&\AMConf{\AMIns{le}:c}{v_1:v_2:e}{s} \AMArrow
 \AMConf{c}{v_1 \leq_{SE} v_2:e}{s}& \text{ if } v_1, v_2 \in \:& \sign \\
&\AMConf{\AMIns{and}:c}{t_1:t_2:e}{s} \AMArrow
 \AMConf{c}{t_1 \land_{SE} t_2:e}{s}& \text{ if } t_1, t_2 \in \:& \TT \\
&\AMConf{\AMIns{neg}:c}{t_1:t_2:e}{s} \AMArrow
 \AMConf{c}{t_1 \: \neg_{SE}\: t_2:e}{s}& \text{ if } t_1, t_2 \in \:& \TT \\
&\AMConf{\AMIns{fetch-}x:c}{e}{s} \AMArrow 
 \AMConf{c}{(s\;x):e}{s} \\
&\AMConf{\AMIns{store-}x:c}{v:e}{s} \AMArrow
 \begin{cases}
    \AMConf{c}{e}{s[x \mapsto v]} & \text{ if } v \sqsubseteq_{SE} Z \\
    \AMConf{c}{e}{\hat{s}} & \text{ if } ERR_A \sqsubseteq_{SE} v \\
    \AMConf{c}{e}{s[x \mapsto v \underset{SE}{\sqcap} Z]} & 
    \text{ if } Z \sqsubseteq_{SE} v \\
 \end{cases}& \\
&\AMConf{\AMIns{noop}:c}{e}{s} \AMArrow 
 \AMConf{c}{e}{s} \\
&\AMConf{\AMIns{branch}(c_1,c_2):c}{v:e}{s} \AMArrow
 \begin{cases}
    \AMConf{c_1:c}{e}{s} & \text{ if } TT \sqsubseteq_{TE} v \\
    \AMConf{c_2:c}{e}{s} & \text{ if } FF \sqsubseteq_{TE} v \\
    \AMConf{c}{e}{\hat{s}} & \text{ if } ERR_B \sqsubseteq_{TE} v \\
 \end{cases}& \\
&\AMConf{\AMIns{loop}(c_1,c_2):c}{e}{s} \AMArrow 
 \AMConf{c_1:\AMIns{branch}(c_2:\AMIns{loop}(c_1, c_2), \AMIns{noop}):c}{e}{s}\\
&\AMConf{\AMIns{try}(c_1, c_2):c}{e}{s} \AMArrow
 \AMConf{c_1:\AMIns{catch}(c_2):c}{e}{s}& \\
&\AMConf{\AMIns{catch}(c_1):c}{e}{s} \AMArrow
 \AMConf{c}{e}{s}& \\
&\AMConf{\AMIns{catch}(c_1):c}{e}{\hat{s}} \AMArrow
 \AMConf{c_1:c}{e}{s}& \\
&\AMConf{c_1:c}{e}{\hat{s}} \AMArrow \AMConf{c}{e}{\hat{s}}
\end{align*}
In the rule for \textsc{store}, \(\underset{SE}{\sqcap}\) means the greatest 
lower bound in the \(\sign\) lattice.
Note that, in the rule for \textsc{store}, if \(x\) equals \(Z\), the 
configuration returned from the first and third case will be equivalent.
Therefore the result becomes only one configuration. Also note that
\textsc{store} and \textsc{branch} are the only nondeterministic rules.

\section*{Analysis}
  Under the right conditions the analysis is able to show a number of
  interesting features of the analysed program, such as:

  \begin{description}
    \item[Variables values] Features of the variables values in the
      configuration states throughout the program. E.g.\ a variable may always
      be positive at a control point in the program.
    \item[Exceptional states] Control points where an exception may occur.
    \item[Unreachable code] Control points in the program that cannot be
      reached in the given context.
    \item[Unneeded try-catch constructs] For example if an exceptional state
      impossibly can occur in a try block.
    \item[Termination] The analysis can tell if the program will terminate
      normally if it terminates or if the termination may be exceptional. In
      some cases the analysis is able to tell that the program won't terminate.
  \end{description}

  \subsection*{Examples}
    Below follows a number of results obtained from the analysis.

    \subsubsection*{Infinite looping}
      In listing \ref{lst:inf} the analysis is able to capture that the program
      won't terminate. If the line $y := y - 1$ had been interchanged with $y
      := y + 1$ though, the analysis wouldn't have catched this although the
      program wouldn't have terminated in this case either.

      \lstinputlisting[
        caption={Analysis of a non-terminating program},
        label=lst:inf]{analysis/infinite.analysis}

    \subsubsection*{Catch}
      In listing \ref{lst:catch-1} the analysis comes to the result that the
      catch block is unneeded. Correspondingly, in listing \ref{lst:catch-2}
      the analysis discovers that the program will always go through the catch
      block.

      \lstinputlisting[
        caption={Analysis of a program with an unneeded catch block},
        label=lst:catch-1]{analysis/unneeded_catch.analysis}

      \lstinputlisting[
        caption={Analysis of a program where the catch block is always reached},
        label=lst:catch-2]{analysis/always_catch.analysis}

\section*{Discussion}
  Alas, the analysis is not very good for all but very simple programs. For
  example, in the program given below $x$ is clearly evaluated to $0$, but the
  analysis will evaluate it to a non error value -- that is either positive,
  negative or zero which is pretty much useless for further analysis.

  \[
    x := 5-5
  \]

  A better analysis for example given above is pretty easy to incorporate in
  the analyzer. For a better analysis it would be useful with more discrete
  possible values. Many loops operate over quite small ranges of integer
  values, e.g.\ $x \in [0,255]$. It would be quite easy to add more discrete
  values to the analyzer although the analyzing process would take more time.

  Another example of a feature that's missing in the analyzer is that simple
  conditions that always hold are not captured. For example, in listing
  \ref{lst:invariant}, $x > y$ always holds. Thus the else block will never be
  executed, but the analyzer will not capture this.

  \begin{lstlisting}[
      caption={A program where $y > x$ always holds},
      label=lst:invariant,
      gobble=4
    ]
    x := 0 ;
    y := 2 ;
    i := 0 ;

    (while i <= 10 do
      x := x + 1 ;
      y := y + 1 ;
      i := i + 1
    ) ;

    if x <= y then
      x := 0
    else
      x := 1
  \end{lstlisting}

\end{document}

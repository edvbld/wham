\documentclass[11pt,oneside,a4paper]{article}
\usepackage{amsmath}
\usepackage{amsfonts}
\usepackage{stmaryrd}
\usepackage{trfrac}

\newenvironment{changemargin}[2]{%
\begin{list}{}{%
\setlength{\topsep}{0pt}%
\setlength{\leftmargin}{#1}% \setlength{\rightmargin}{#2}%
\setlength{\listparindent}{\parindent}%
\setlength{\itemindent}{\parindent}%
\setlength{\parsep}{\parskip}%
}%
\item[]}{\end{list}}

\newcommand{\SExp}[2]{\mathcal{#1}\llbracket #2 \rrbracket}
\newcommand{\AExp}[2]{\SExp{A}{#1}(#2)}
\newcommand{\BExp}[2]{\SExp{B}{#1}(#2)}
\newcommand{\AMIns}[1]{\textsc{#1}}
\newcommand{\AMConf}[3]{\langle #1, #2, #3 \rangle}
\newcommand{\AMArrow}{\: &\triangleright&\:}
\newcommand{\NSConf}[2]{\langle #1, #2 \rangle}
\newcommand{\NSProd}[3]{\NSConf{#1}{#2} \rightarrow #3}
\newcommand{\NSRule}[3]{[\text{#1}_{ns}^{#2}] \quad \quad & #3}
\newcommand{\tdots}{\text{\dots}}
\newcommand{\smap}[2]{[#1 \mapsto \textbf{#2}]}
  
\begin{document}
\title{Report for Lab Assignment 1, DD2457 Program Semantics and Analysis}
\author{Erik Helin}
\date{\today}
\maketitle
\newpage

\section*{New Rules}
The following new rules have been added to the various semantics:
\subsection*{Natural semantics}
\begingroup
\addtolength{\jot}{.5em}
\begin{align*}
\NSRule{try}{1}{\NSProd{\texttt{try } S_1 \texttt{ catch } S_2}{\hat{s}}{\hat{s}}} \\
\NSRule{try}{2}{\frac{\NSProd{S_1}{s}{s^{'}}}
                     {\NSProd{\texttt{try } S_1 \texttt{ catch } S_2}{s}{s^{'}}}} \\
[\text{try}_{ns}^3] \quad \quad &
\frac{\langle S_1,s\rangle  \rightarrow \hat{s}, \: \langle S_2, \hat{s}\rangle  \rightarrow s^{'}}
     {\langle \texttt{try}\: S_1 \: \texttt{catch} \: S_2, s\rangle  \rightarrow s^{'}} \\
[\text{ass}_{ns}^2] \quad \quad &
\langle x := a, \hat{s}\rangle  \rightarrow \hat{s} \\
[\text{ass}_{ns}^3] \quad \quad &
\langle x := a, s\rangle  \rightarrow \hat{s} \quad 
\text{if} \: \mathcal{A} \llbracket a \rrbracket s = \bot \\
[\text{skip}_{ns}^2] \quad \quad &
\langle \texttt{skip}, \hat{s}\rangle  \rightarrow \hat{s} \\
[\text{comp}_{ns}^2] \quad \quad &
\langle S_1;S_2, \hat{s}\rangle  \rightarrow \hat{s} \\
[\text{comp}_{ns}^3] \quad \quad &
\frac{\langle S_1, s\rangle  \rightarrow \hat{s}}{\langle S_1;S_2, s\rangle  \rightarrow \hat{s}} \\
[\text{comp}_{ns}^4] \quad \quad &
\frac{\langle S_1, s\rangle  \rightarrow s^{'}, \langle S_2, s^{'}\rangle  \rightarrow \hat{s}}
{\langle S_1;S_2, s\rangle  \rightarrow \hat{s}} \\
[\text{if}_{ns}^{\:3}] \quad \quad & \langle \texttt{if}\:b\: \texttt{then}\:  S_1 \: 
\texttt{else} \: S_2, \hat{s}\rangle  \rightarrow \hat{s} \\
[\text{if}_{ns}^{\:4}] \quad \quad & \frac{\langle S_1, s\rangle  \rightarrow \hat{s}}
{ \langle \texttt{if}\:b\: \texttt{then}\:  S_1 \: \texttt{else} \: S_2, s\rangle  
\rightarrow \hat{s}} \quad \text{if} \: \mathcal{B} \llbracket b \rrbracket s
= \textbf{tt}\\
[\text{if}_{ns}^{\:5}] \quad \quad & \frac{\langle S_2, s\rangle  \rightarrow \hat{s}}
{ \langle \texttt{if}\:b\: \texttt{then}\:  S_1 \: \texttt{else} \: S_2, s\rangle  
\rightarrow \hat{s}} \quad \text{if} \: \mathcal{B} \llbracket b \rrbracket s
= \textbf{ff}\\
[\text{if}_{ns}^{\:6}] \quad \quad & \langle \texttt{if}\:b\: \texttt{then}\:  S_1 \: 
\texttt{else} \: S_2, s\rangle  \rightarrow \hat{s} \quad 
\text{if} \: \mathcal{B} \llbracket b \rrbracket s
= \bot \\
[\text{while}_{ns}^3] \quad \quad & \langle \texttt{while}\: b \:\texttt{do}\: S, 
\hat{s}\rangle  \rightarrow \hat{s} \\
[\text{while}_{ns}^4] \quad \quad & 
\langle \texttt{while}\: b \:\texttt{do}\: S, s\rangle  \rightarrow \hat{s} 
\quad \text{if} \: \mathcal{B} \llbracket b \rrbracket s = \bot \\
[\text{while}_{ns}^5] \quad \quad & 
\frac{\langle S,s\rangle  \rightarrow s^{'}, \langle \texttt{while}\: b \: \texttt{do} \: S, s^{'}\rangle 
\rightarrow \hat{s}}
{\langle \texttt{while}\: b \: \texttt{do}\: S, s\rangle  \rightarrow \hat{s}} 
\quad \text{if} \: \mathcal{B} \llbracket b \rrbracket s = \textbf{tt} \\
\end{align*}
\endgroup

\subsection*{Arithmetic expressions}
\begin{align*}
&\mathcal{A}\llbracket a_1 / a_2 \rrbracket(s) &=&\: \mathcal{A}\llbracket a_1
\rrbracket(s) / \mathcal{A}\llbracket a_2 \rrbracket(s)& &\text{ if }
\mathcal{A}\llbracket a_2 \rrbracket(s) \neq 0 \\
&\AExp{a_1 / a_2}{s} &=&\: \bot& &\text{ if } 
\AExp{a_2}{s} = 0 \text{ or } \AExp{a_1}{s} = \bot 
\text{ or } \AExp{a_2}{s} = \bot \\
&\AExp{a_1 + a_2}{s} &=&\: \bot &\: &\text{ if } \AExp{a_1}{s} = \bot 
\text{ or } \AExp{a_2}{s} = \bot \\
&\AExp{a_1 - a_2}{s} &=&\: \bot &\: &\text{ if } \AExp{a_1}{s} = \bot 
\text{ or } \AExp{a_2}{s} = \bot \\
&\AExp{a_1 * a_2}{s} &=&\: \bot &\: &\text{ if } \AExp{a_1}{s} = \bot 
\text{ or } \AExp{a_2}{s} = \bot \\
\end{align*}

\subsection*{Boolean expressions}
\begin{align*} 
\BExp{a_1 = a_ 2}{s} &=
    \begin{cases}
        \textbf{tt} & \text{if } \AExp{a_1}{s} = \AExp{a_2}{s} \\
        \textbf{ff} & \text{if } \AExp{a_1}{s} \neq \AExp{a_2}{s} \\
        \bot & \text{if } \AExp{a_1}{s} = \bot \text{ or } 
                          \AExp{a_2}{s} = \bot \\
    \end{cases} \\
\mathcal{B}\llbracket a_1 \leq a_2 \rrbracket(s) &= 
    \begin{cases}
        \textbf{tt} & \text{if } \mathcal{A}\llbracket a_1 \rrbracket(s) \leq
        \mathcal{A}\llbracket a_2 \rrbracket(s) \\
        \textbf{ff} & \text{if } \mathcal{A}\llbracket a_1 \rrbracket(s) >
        \mathcal{A}\llbracket a_2 \rrbracket(s)  \\
        \bot & \text{if } \mathcal{A}\llbracket a_1 \rrbracket(s) = \bot 
                          \text{ or }
                          \mathcal{A} \llbracket a_2 \rrbracket (s) = \bot
    \end{cases} \\
\mathcal{B}\llbracket \neg\, b \rrbracket(s) &= 
    \begin{cases}
        \textbf{tt} & \text{if } \mathcal{B}\llbracket b \rrbracket(s) =
        \textbf{ff} \\
        \textbf{ff} & \text{if } \mathcal{B}\llbracket b \rrbracket(s) =
        \textbf{tt} \\
        \bot & \text{if }\mathcal{B}\llbracket b \rrbracket(s) = \bot
    \end{cases} \\
\mathcal{B}\llbracket b_1 \land b_2 \rrbracket(s) &= 
    \begin{cases}
        \textbf{tt} & \text{if } \mathcal{B}\llbracket b_1 \rrbracket(s) =
        \textbf{tt} \text{ and } \mathcal{B}\llbracket b_2 \rrbracket(s) =
        \textbf{tt} \\
        \textbf{ff} & \text{if } \mathcal{B}\llbracket b_1 \rrbracket(s) =
        \textbf{ff} \text{ or } \mathcal{B}\llbracket b_2 \rrbracket(s) =
        \textbf{ff} \\
        \bot & \text{if }\mathcal{B}\llbracket b_1 \rrbracket(s) = \bot
                         \text{ or }
                         \mathcal{B}\llbracket b_2 \rrbracket(s) = \bot
    \end{cases}
\end{align*}

\subsection*{AM instructions}
\begin{align*}
\AMIns{div}\: |\: \AMIns{try}\:(c_1,c_2)\: |\: \AMIns{catch}\:(c)
\end{align*}

\subsection*{AM translations}
\begin{align*}
\SExp{CA}{a_1 / a_2} &= 
    \SExp{CA}{a_2}:\SExp{CA}{a_1}:\AMIns{div} \\
\SExp{CS}{\texttt{try } S_1  \texttt{ catch } S_2} &= 
    \AMIns{try}(\SExp{CS}{S_1}, \SExp{CS}{S_2})
\end{align*}

\subsection*{AM Structural Operational Rules}
\begingroup
\addtolength{\jot}{.5em}
\begin{align*}
&\AMConf{\AMIns{div}:c}{n_1:n_2:e}{s} \AMArrow
\AMConf{c}{(n_1 / n_2):e}{s}& \text{ if } 
n_1, n_2 \in \mathbb{Z}, n_2 \neq 0 \\
&\AMConf{\AMIns{div}:c}{n_1:n_2:e}{s} \AMArrow
\AMConf{c}{\bot:e}{\hat{s}}& \text{ if }
n_1, n_2 \in \mathbb{Z}, n_2 = 0 \\
&\AMConf{\AMIns{try}(c_1, c_2):c}{e}{s} \AMArrow
\AMConf{c_1:\AMIns{catch}(c_2):c}{e}{s}& \\
&\AMConf{\AMIns{catch}(c_1):c}{e}{s} \AMArrow
\AMConf{c}{e}{s}& \\
&\AMConf{\AMIns{catch}(c_1):c}{\bot:e}{\hat{s}} \AMArrow
\AMConf{c_1:c}{e}{s}& \\
&\AMConf{c_1:c}{e}{\hat{s}} \AMArrow \AMConf{c}{e}{\hat{s}}
\end{align*}
\endgroup

\subsection*{Derivation tree}
Here follows the derivation tree of the program
\begin{verbatim}
x := 7; try x := x - 7; x := 7 / x; x := x + 7 catch x := x - 7
\end{verbatim}
\begin{changemargin}{-2.5cm}{\rightmargin}
\[
\trfrac[\(\text{comp}_{ns}\)]{
\trfrac[\(\text{ass}_{ns}\)]{s_0[x \mapsto 7]}{<x := 7, s_0> \rightarrow s_2}
\trfrac[\(\text{try}^3_{ns}\)]{
\trfrac[\(\text{comp}_{ns}\)]{T_1}
{<x := x - 7; x := 7 / x; x := x + 7, s_2> \rightarrow \hat{s}_3} 
\trfrac[\(\text{ass}_{ns}\)]{s_3[x \mapsto s_3(x) - 7\:]}
{<x := x - 7, \hat{s}_3> \rightarrow s_1}}
{<\textbf{try}\: x := x - 7; x := 7 / x; x := x + 7 \: \textbf{catch}\:
x := x - 7, s_2> \rightarrow s_1}}
{<x := 7; \textbf{try}\: x := x - 7; x := 7 / x; x := x + 7 \: \textbf{catch}\:
x := x - 7, s_0> \rightarrow s_1}
\]
\end{changemargin}
\[
\trfrac[]{
\trfrac[\(\text{ass}_{ns}\)]{s_2[x \mapsto s_2(x) - 7]}
{<x := x - 7, s_2> \rightarrow s_3}
\trfrac[\(\text{comp}_{ns}\)]{
\trfrac[\(\text{ass}^3_{ns}\)]{\hat{s}_3 = (s_3, \bot)}
{<x := 7/x, s_3> \rightarrow \hat{s}_3}
<x := x + 7, \hat{s_3}> \rightarrow \hat{s}_3}
{<x := 7 / x; x := x + 7, s_3> \rightarrow \hat{s}_3}
}{T_1}
\]

where
\begin{align*}
s_2 &= (s_0[x \mapsto 7], \top) \\
s_3 &= (s_2[x \mapsto s_2(x) - 7], \top) = (s_0[x \mapsto 0], \top) \\
\hat{s}_3 &= (s_3, \bot) = (s_0[x \mapsto 0], \bot) \\
s_1 &= (s_3[x \mapsto s_3(x) - 7], \top) = (s_0[x \mapsto s_0(x) - 7], \top) =
(s_0[x \mapsto -7], \top)
\end{align*}

\subsection*{Execution of sample}
Here follows the execution of the program from an arbitrary 
state \(s\).
\begin{verbatim}
x := 7; try x := x - 7; x := 7 / x; x := x + 7 catch x := x - 7
\end{verbatim}
First, the program is translated
\begin{changemargin}{-2.9cm}{\rightmargin}
\begin{align*}
&\SExp{CS}{x := 7; \textbf{try } x := x - 7; x := 7 / x; x := x + 7 
\textbf{ catch } x := x - 7}  \\
=\: &\SExp{CS}{x := 7}:\SExp{CS}{\textbf{try } x := x - 7; x := 7 / x; x := x + 7 
\textbf{ catch } x := x - 7} \\
=\: &\SExp{CA}{7}:\AMIns{store-x}:\SExp{CS}{\textbf{try } x := x - 7; x := 7 / x; x := x + 7 
\textbf{ catch } x := x - 7} \\
=\: &\AMIns{push-7}:\AMIns{store-x}:\SExp{CS}{\textbf{try } x := x - 7; x := 7 / x; x := x + 7 
\textbf{ catch } x := x - 7}  \\
=\: &\AMIns{push-7}:\AMIns{store-x}:
\AMIns{try}(\SExp{CS}{x := x - 7; x := 7 / x; x := x + 7}, 
\SExp{CS}{x := x - 7}) \\
=\: &\AMIns{push-7}:\AMIns{store-x}:
\AMIns{try}(\SExp{CS}{x := x - 7}:\SExp{CS}{x := 7 / x; x := x + 7},
\SExp{CS}{x := x - 7}) \\
=\: &\AMIns{push-7}:\AMIns{store-x}:
\AMIns{try}(\SExp{CA}{x - 7}:\AMIns{store-x}:\SExp{CS}{x := 7 / x; x := x + 7},
\SExp{CS}{x := x - 7}) \\
=\: &\AMIns{push-7}:\AMIns{store-x}:
\AMIns{try}(\SExp{CA}{7}:\SExp{CA}{x}:\AMIns{sub}:\AMIns{store-x}:\SExp{CS}{x := 7 / x; x := x + 7},
\SExp{CS}{x := x - 7}) \\
=\: &\AMIns{push-7}:\AMIns{store-x}:
\AMIns{try}(\AMIns{push-7}:\SExp{CA}{x}:\AMIns{sub}:\AMIns{store-x}:\SExp{CS}{x := 7 / x; x := x + 7},
\SExp{CS}{x := x - 7}) \\
=\: &\AMIns{push-7}:\AMIns{store-x}:
\AMIns{try}(\AMIns{push-7}:\AMIns{fetch-x}:\AMIns{sub}:\AMIns{store-x}:\SExp{CS}{x := 7 / x; x := x + 7},
\SExp{CS}{x := x - 7}) \\
=\: &\AMIns{push-7}:\AMIns{store-x}:
\AMIns{try}(\AMIns{push-7}:\AMIns{fetch-x}:\AMIns{sub}:\AMIns{store-x}:\SExp{CS}{x
:= 7 / x}:\SExp{CS}{x := x + 7},
\SExp{CS}{x := x - 7}) \\
=\: &\AMIns{push-7}:\AMIns{store-x}: \\
&\AMIns{try}(\AMIns{push-7}:\AMIns{fetch-x}:\AMIns{sub}:\AMIns{store-x}:
\SExp{CA}{7 / x}:\AMIns{store-x}:\SExp{CS}{x := x + 7},
\SExp{CS}{x := x - 7}) \\
=\: &\AMIns{push-7}:\AMIns{store-x}: \\
&\AMIns{try}(\AMIns{push-7}:\AMIns{fetch-x}:\AMIns{sub}:\AMIns{store-x}:
\SExp{CA}{x}:\SExp{CA}{7}:\AMIns{div}:\AMIns{store-x}:\SExp{CS}{x := x + 7},
\SExp{CS}{x := x - 7}) \\
=\: &\AMIns{push-7}:\AMIns{store-x}: \\
&\AMIns{try}(\AMIns{push-7}:\AMIns{fetch-x}:\AMIns{sub}:\AMIns{store-x}:
\SExp{CA}{x}:\AMIns{push-7}:\AMIns{div}:\AMIns{store-x}:\SExp{CS}{x := x + 7},
\SExp{CS}{x := x - 7}) \\
=\: &\AMIns{push-7}:\AMIns{store-x}: \\
&\AMIns{try}(\AMIns{push-7}:\AMIns{fetch-x}:\AMIns{sub}:\AMIns{store-x}:
\AMIns{fetch-x}:\AMIns{push-7}:\AMIns{div}:\AMIns{store-x}:\SExp{CS}{x := x + 7},
\SExp{CS}{x := x - 7}) \\
=\: &\AMIns{push-7}:\AMIns{store-x}: \\
&\AMIns{try}(\AMIns{push-7}:\AMIns{fetch-x}:\AMIns{sub}:\AMIns{store-x}:
\AMIns{fetch-x}:\AMIns{push-7}:\AMIns{div}:\AMIns{store-x}:
\SExp{CA}{x + 7}:\AMIns{store-x}, \\
& \quad \quad \: \SExp{CS}{x := x - 7}) \\
=\: &\AMIns{push-7}:\AMIns{store-x}: \\
&\AMIns{try}(\AMIns{push-7}:\AMIns{fetch-x}:\AMIns{sub}:\AMIns{store-x}:
\AMIns{fetch-x}:\AMIns{push-7}:\AMIns{div}:\AMIns{store-x}:
\AMIns{push-7}:\AMIns{fetch-x}:\AMIns{add}:\AMIns{store-x}, \\
& \quad \quad \: \SExp{CS}{x := x - 7}) \\
=\: &\AMIns{push-7}:\AMIns{store-x}: \\
&\AMIns{try}(\AMIns{push-7}:\AMIns{fetch-x}:\AMIns{sub}:\AMIns{store-x}:
\AMIns{fetch-x}:\AMIns{push-7}:\AMIns{div}:\AMIns{store-x}:
\AMIns{push-7}:\AMIns{fetch-x}:\AMIns{add}:\AMIns{store-x}, \\
& \quad \quad \: \SExp{CA}{x - 7}:\AMIns{store-x}) \\
=\: &\AMIns{push-7}:\AMIns{store-x}: \\
&\AMIns{try}(\AMIns{push-7}:\AMIns{fetch-x}:\AMIns{sub}:\AMIns{store-x}:
\AMIns{fetch-x}:\AMIns{push-7}:\AMIns{div}:\AMIns{store-x}:
\AMIns{push-7}:\AMIns{fetch-x}:\AMIns{add}:\AMIns{store-x}, \\
& \quad \quad \: \AMIns{push-7}:\AMIns{fetch-x}:\AMIns{sub}:\AMIns{store-x}) \\
\end{align*}
\end{changemargin}

Then the translated program is executed

\begin{align*}
&\AMConf{\AMIns{push-7}:\AMIns{store-x}:\AMIns{try}(\tdots)}{\epsilon}{s}
\\
\triangleright \: &\AMConf{\AMIns{store-x}:\AMIns{try}(\tdots)}{\textbf{7}}{s}
\\
\triangleright \: &\AMConf{\AMIns{try}(\tdots)}{\epsilon} {s\smap{x}{7}} \\
\triangleright \: &
\AMConf{\AMIns{push-7}:\AMIns{fetch-x}:\AMIns{sub}:\AMIns{store-x}:
        \tdots:\AMIns{catch}(\AMIns{push-7}:\AMIns{fetch-x}:\AMIns{sub}:
                             \AMIns{store-x})}{\epsilon}{s\smap{x}{7}} \\
\triangleright \: &
\AMConf{\AMIns{fetch-x}:\AMIns{sub}:\AMIns{store-x}:
        \tdots:\AMIns{catch}(\AMIns{push-7}:\AMIns{fetch-x}:\AMIns{sub}:
                             \AMIns{store-x})}{\textbf{7}}{s\smap{x}{7}} \\
\triangleright \: &
\AMConf{\AMIns{sub}:\AMIns{store-x}:
        \tdots:\AMIns{catch}(\AMIns{push-7}:\AMIns{fetch-x}:\AMIns{sub}:
                             \AMIns{store-x})}{\textbf{7}:\textbf{7}}{s\smap{x}{7}} \\
\triangleright \: &
\AMConf{\AMIns{store-x}:
        \tdots:\AMIns{catch}(\AMIns{push-7}:\AMIns{fetch-x}:\AMIns{sub}:
                             \AMIns{store-x})}{\textbf{0}}{s\smap{x}{7}} \\
\triangleright \: &
\AMConf{\AMIns{fetch-x}:\AMIns{push-7}:\AMIns{div}:\AMIns{store-x}:
\tdots:\AMIns{catch}(\AMIns{push-7}:\AMIns{fetch-x}:\AMIns{sub}:
                             \AMIns{store-x})}{\epsilon}{s\smap{x}{0}} \\
\triangleright \: &
\AMConf{\AMIns{push-7}:\AMIns{div}:\AMIns{store-x}:
\tdots:\AMIns{catch}(\AMIns{push-7}:\AMIns{fetch-x}:\AMIns{sub}:
                     \AMIns{store-x})}{\textbf{0}}{s\smap{x}{0}} \\
\triangleright \: &
\AMConf{\AMIns{div}:\AMIns{store-x}:
\tdots:\AMIns{catch}(\AMIns{push-7}:\AMIns{fetch-x}:\AMIns{sub}:
                     \AMIns{store-x})}{\textbf{7}:\textbf{0}}{s\smap{x}{0}} \\
\triangleright \: &
\AMConf{\AMIns{store-x}:
\tdots:\AMIns{catch}(\AMIns{push-7}:\AMIns{fetch-x}:\AMIns{sub}:
                     \AMIns{store-x})}{\bot}{\hat{s}\smap{x}{0}} \\
\triangleright \: &
\AMConf{\tdots:\AMIns{catch}(\AMIns{push-7}:\AMIns{fetch-x}:\AMIns{sub}:
                     \AMIns{store-x})}{\bot}{\hat{s}\smap{x}{0}} \\
\triangleright \: &
\AMConf{\AMIns{catch}(\AMIns{push-7}:\AMIns{fetch-x}:\AMIns{sub}:
                     \AMIns{store-x})}{\bot}{\hat{s}\smap{x}{0}} \\
\triangleright \: &
\AMConf{\AMIns{push-7}:\AMIns{fetch-x}:\AMIns{sub}:
                     \AMIns{store-x}}{\epsilon}{s\smap{x}{0}} \\
\triangleright \: &
\AMConf{\AMIns{fetch-x}:\AMIns{sub}:\AMIns{store-x}}{\textbf{7}}{s\smap{x}{0}} \\
\triangleright \: &
\AMConf{\AMIns{sub}:\AMIns{store-x}}{\textbf{0}:\textbf{7}}{s\smap{x}{0}} \\
\triangleright \: &
\AMConf{\AMIns{store-x}}{\textbf{-7}}{s\smap{x}{0}} \\
\triangleright \: &
\AMConf{\epsilon}{\epsilon}{s\smap{x}{-7}} \\
\end{align*}

\subsection*{Implementation}
The programming part of the lab was done in the programming language Haskell. 
For the parser, the monadic parser combinator library parsec was used.
The result was four modules:
\begin{description}
\item[\texttt{Parser.hs}] containing all the parser code
\item[\texttt{Translator.hs}] for translating the AST to AM code
\item[\texttt{Interpreter.hs}] for executing the AM code
\item[\texttt{Main.hs}] for invoking the program
\end{description}
To use the interpreter, run \texttt{./wham <name-of-file>}. To start the 
program from a used defined state where \(x \mapsto 5\) and \(y \mapsto 10\), 
run 
\begin{verbatim} 
./wham -s '[("x", 5),("y", 10)]' <name-of-file>
\end{verbatim}
To print debug output, use the \texttt{-d} flag.

\subsection*{Argument of correctness}
The modified version of lemma 4.18 used in class still holds, 
but the lemma have to be adjusted a little bit to take the 
extended states into account:

{\bf Lemma 4.18*} 
\begin{align*}
\forall a \in \text{AExp}.\forall c \in \text{Code}.\forall e \in \text{Stack}.
\forall s \in \text{EState}. \\
\langle\mathcal{CA}\llbracket a \rrbracket:c, e, s \rangle \triangleright^+ 
\langle c, \mathcal{A}\llbracket a \rrbracket:e, s \rangle
\end{align*}

{\bf Proof}
The proof is the same as before (the one we did in class), 
but a new case is added for division
\begin{align*}
[a \equiv a_1 / a_2]&\:\:\:\:\:\: \langle \mathcal{CA}\llbracket a_1 / a_2 
\rrbracket:c,
e, s \rangle \\
&= \langle \mathcal{CA}\llbracket a_2 \rrbracket:\mathcal{CA} \llbracket a_1
\rrbracket:\textsc{DIV}:c, e, s\rangle & \{\text{def}\:\mathcal{CA}\} \\
&\triangleright^+ \langle \mathcal{CA} \llbracket a_1 \rrbracket:\textsc{DIV}:c,
\mathcal{A}\llbracket a_2 \rrbracket(s):e, s \rangle & \{\text{ind. hyp.}\} \\
&\triangleright^+ \langle \textsc{DIV}:c,
\mathcal{A}\llbracket a_1 \rrbracket(s):\mathcal{A}\llbracket a_2 
\rrbracket(s):e, 
s \rangle & \{\text{ind. hyp.}\} \\
&\triangleright \langle c, (\mathcal{A}\llbracket a_1 \rrbracket(s) /
\mathcal{A}\llbracket a_2 \rrbracket(s)):e, s \rangle & \{\text{def}\: 
\triangleright\} \\
\end{align*}

Lemma 4.19 would have to be adjusted as well to take extended states into
account.

The biggest change that is required in Lemma 4.21 and Lemma 4.22 is to allow
the execution of code without requiring an empty stack. Lemma 4.21 would
have to be changed to
\begin{align*}
&\forall S \in Stm.\forall s, s' \in EState. \\
&\NSProd{S}{s}{s'} \implies \AMConf{\SExp{CS}{S}}{e}{s}
\triangleright^{*} \AMConf{\epsilon}{e'}{s'}
\end{align*}
However, due to Exercise 4.4, the structure of the proof will stay the same.
One would of course have to add cases for the exceptional states, e.g. when 
\(\NSProd{S_1;S_2}{s}{\hat{s}}\) due to \(\NSProd{S_1}{s}{\hat{s}}\) and also 
add a proof for the case when \(S \equiv \textbf{try } S_1 \textbf{ catch }
S_2\).

Lemma 4.22 would have to be changed to
\begin{align*}
&\forall S \in Stm.\forall s, s' \in EState. \\
&\AMConf{\SExp{CS}{S}}{e}{s} \triangleright^k \AMConf{\epsilon}{e'}{s'} \implies
\NSProd{S}{s}{s'}
\end{align*}
Again, the proof would have to be adjusted for allowing the stack to be
non-empty when the code is empty. This should not alter the structure of the
proof too much, once again due to Lemma 4.4. Also cases has to be added 
for exceptional states and also a case \(S \equiv \textbf{try } S_1 \textbf{ catch }
S_2\).


\subsection*{Examples}
The implementation was tested with all the examples in the folder 
\texttt{/info/semant11/lab1/samples} all well as the following examples.
\subsubsection*{Example 1}
\begin{description}
\item[Description] Exception in the boolean condition of an if statement 
\item[Result] \((s[x \mapsto 0], \bot)\) for all initial states \(s\)
\item[Code]
\begin{verbatim}
x := 0; if 7 / 0 <= 7 then x := 1 else x := 2; x := x - 2 
\end{verbatim}
\end{description}

\subsubsection*{Example 2}
\begin{description}
\item[Description] Exception in the boolean condition of a while loop
\item[Result] \((s[x \mapsto 1], \bot)\) for all initial states \(s\)
\item[Code]
\begin{verbatim}
x := 1; while x <= x / 0 do x := x + 1; x := x - 1
\end{verbatim}
\end{description}

\subsubsection*{Example 3}
\begin{description}
\item[Description] An exception before a try/catch statement
\item[Result] \((s, \bot)\) for all initial states \(s\)
\item[Code]
\begin{verbatim}
x := 7 / 0; try x := 1 catch x := 2; x := 3
\end{verbatim}
\end{description}

\subsubsection*{Example 4}
\begin{description}
\item[Description] An exception in a catch statement of a nested try/catch
statement
\item[Result] \((s[x \mapsto 5][y \mapsto 10], \top)\) for all initial states
\(s\)
\item[Code] \hfill
\begin{verbatim}
x := 0;
y := 0;
try
    try
        x := 0;
        y := 5 / x
    catch
        x := 5;
        y := 0;
        x := 5 / y
catch
    y := x + x
\end{verbatim}
\end{description}
\end{document}
